\section{Introduction}

In 1952, the \textit{Met Office}, i.e., the UK's national weather service, successfully completed its first numerical weather prediction forecast on a 12 \(\times\) 8 grid with grid point spacing of \SI{260}{\km} \parencite{MetOffice}. Since then, weather forecasts have improved tremendously, not least because of an increase in grid resolution. This development has been made possible by the continued growth of available computational power. While, on the one hand, increased grid resolution undoubtedly improved the numerical weather forecast, on the other hand, it also increased the need for adequate computational resources. The CFL criterion introduces an upper limit for the time step proportional to the grid point separation---a high grid resolution thus demands a short time step, which is computationally more expensive. Therefore, it is of interest to understand the influence of grid resolution on numerical weather prediction in order to find the best compromise of grid point spacing in terms of computing time and forecast quality. 
% hier fehlt irgendwie noch ein Satz, dass mein Projekt natürlich zu klein ist um diese Frage wirklich zu beantworten
In this study, a regional weather model calculates one-month forecasts on three different spatial resolutions for the region of Europe. Then, the change in precipitation patterns due to varying grid size is analysed. Generally speaking, precipitation plays a major role in weather forecasts, hence, it seems natural to use this variable in order to compare grid resolution. On top, as a diagnostic variable, it is based on multiple direct model predictions and captures the effects of resolution-dependent internal model dynamics.