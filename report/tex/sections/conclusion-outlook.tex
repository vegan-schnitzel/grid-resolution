\section{Conclusion}

% When localisierung wichtig ist, kleine Gittergröße, aber mean precipitation bleibt auch schon bei geringer auflösung gleich.

% hihe auflösung definitiv bei extremen besser (aber keine richtige convection), aber niedrige auflösung genügt um die strukturen wiederzugeben und sobald gemittelt wird, ist alles okay!

% outlook: mit alterntivem convection scheme, mehr tage, mehr daten, statistik,,, ensembles um unsicherheit zu klassifizeiren!

This study provides details about the influence of different grid resolutions on precipitation forecasts using the CCLM regional climate model. High grid resolution (expectedly) improves localization of the precipitation events. Differences in precipitation intensity are clearly visible on maps, but almost disappear when averaging over a sufficiently large region. Precise localization of precipitation events implies higher number of grid boxes without precipitation, which results in similar precipitation signals when averaging, regardless of the grid resolution. The remaining differences in precipitation for changing grid sizes are mainly found in grid-scale precipitation, although this might change if a deep convection scheme is enabled in the model runs. To summarize, if precise forecasts of intensity and location are of interest, it is vital to increase the grid resolution, however, low grid resolutions can already capture precipitation structures and become increasingly useful, if data availability allows averaging or other applications of simple statistics. These qualitative results could be extended by running ensemble forecasts with perturbed initial and boundary conditions to get an idea of the underlying uncertainty. Additionally, the expected increase in the ratio of convective to grid-scale precipitation in the summer months (JJA) could complement the results if analysis is repeated for such different atmospheric conditions.